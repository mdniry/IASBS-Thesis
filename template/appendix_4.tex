% !TeX root = main.tex
% !TeX spellcheck = fa_IR
\chapter{پرسش‌های متداول و پاسخ‌ها}
اگرچه وجود چنین پیوستی در 
\thesis 
ضروری نیست، اما برای پاسخ به سوالات رایجی که ممکن است هنگام استفاده از این الگو پیش بیاید، این بخش تهیه شده است.


\subsubsection{ارجاع به منابع فارسی در کتاب‌نامه}
چگونه می‌توان به یک منبع فارسی ارجاع داد؟ آیا نمونه‌ای برای این کار وجود دارد؟\\
منابع فارسی مثل مقالات، پایان‌نامه‌ها و رساله‌ها معمولاً با عناوین فارسی و انگلیسی منتشر می‌شوند. برای سهولت ثبت در آرشیوهای بین‌المللی، پیشنهاد می‌شود به عنوان انگلیسی منابع فارسی ارجاع دهید. اگر منبعی فارسی این ویژگی را ندارد و نیاز دارید به عنوان فارسی آن ارجاع دهید می‌توانید از نمونه‌هایی از منابع فارسی که در فایل‌های 
\lr{`MyReferences.bib'} 
و 
\lr{`references.tex'}
گنجانده شده استفاده کنید. با این حال، برای زیبایی و حفظ یکدستی الگو، این نمونه‌ها به‌طور مستقیم نمایش داده نشده‌اند.


\subsubsection{ارجاع به منابع با نام نویسنده و سال انتشار}
در رشته‌هایی که مرسوم است ارجاع‌ها با نام نویسنده و سال باشد، چگونه این کار انجام می‌شود؟\\
برای این نوع ارجاع کافی است الگوی مناسب را در فایل 
\lr{main.tex}
انتخاب کنید. همچنین، چون ذکر نام نویسنده به لاتین در متن زیبایی کمتری دارد، گزینه 
\lr{authorfa}
برای وارد کردن نام نویسندگان به فارسی در فایل 
\lr{`MyReferences.bib'} 
فراهم شده است. این گزینه باید حتماً تکمیل شود تا سامی در متن به شکل صحیح نمایش داده شود.


\subsubsection{عنوان بالای صفحات زوج}
چرا فقط بخشی از عنوان رساله بالای صفحات زوج نمایش داده می‌شود؟ آیا امکان تنظیم این عنوان وجود دارد؟\\
معمولاً عنوان 
\thesis 
طولانی است و نمی‌توان آن را به طور کامل بالای صفحه نمایش داد. بنابراین بخشی از عنوان که بالای صفحات زوج ظاهر می‌شود، همان عنوان کوتاه (\lr{Short Title}) است. مقدار پیش‌فرض آن برابر با بخش اصلی عنوان (خط اول عنوان روی جلد) است. اگر عنوان اصلی به‌طور مناسب انتخاب شده باشد، نیازی به تنظیم جداگانه عنوان کوتاه نیست. در صورت نیاز، می‌توانید با استفاده از دستور
\lr{\texttt{$\backslash$shorttitle\{\dots\}}}
عنوان کوتاه را مشخص کنید.


\subsubsection{اضافه کردن مقاله چاپ شده}
آیا مقاله چاپ شده باید به انتهای رساله اضافه شود؟ اگر بله، چگونه؟\\
براساس الگوی رایج در دانشگاه معمولاً نیازی به اضافه کردن مقاله چاپ شده به رساله نیست، زیرا کیفیت رساله و ارائه آن در جلسه دفاع اهمیت دارد. با این حال، اگر کمیته داوری یا استاد راهنمای شما خواستار اضافه کردن مقاله باشند، می‌توانید با استفاده از دستور:
\begin{flushleft}
\lr{\small\texttt{ $\backslash$includepdf[pages=<\rl{صفحه‌ها}>]\{<\rl{نام فایل}>.pdf\} }}
\end{flushleft}
فایل 
\lr{PDF}
آن را به رساله اضافه کنید.
%\includepdf[pages=10]{xepersian.pdf}


\subsubsection{محل قرارگیری عنوان شکل‌ها}
بهتر است عنوان شکل زیر آن باشد یا کنار آن؟\\
این موضوع به طراحی و نسبت پهنا به ارتفاع شکل‌ها بستگی دارد. برای شکل‌هایی که پهنای آن‌ها بیشتر از ارتفاع است، بهتر است عنوان زیر شکل قرار گیرد. اما اگر ارتفاع شکل بیش از پهنا باشد و فضای سفید قابل توجهی اطراف شکل باقی بماند، ممکن است بهتر باشد عنوان کنار شکل قرار گیرد تا فضای خالی به حداقل برسد. در این الگو هر دو حالت فراهم شده است تا بتوانید مناسب‌ترین روش را انتخاب کنید. 

بهتر است رویه یکسانی را در کل متن دنبال کنید. با توجه به اینکه نتایج خودتان را گزارش می‌کنید، احتمالاً می‌توانید همهٔ نمودارها را با یک نسبت پهنا به ارتفاع آماده کنید. در موارد نادری ممکن است انتخاب عمومی شما درست درنیاید و چرخاندن آن تصویر یا نمودار هم مطلوب به نظر نرسد. در این صورت می‌توانید برای آن مورد یا موارد نادر قاعده کلی را نقض کنید.


\subsubsection{الگوی پایان‌نامه در دانشگاه}
آیا مجبورم از این الگو که در لاتِک تدوین شده استفاده کنم؟\\
نگارش پایان‌نامه یا رساله با لاتِک الزامی نیست. شما می‌توانید از نرم‌افزارهایی نظیر آفیسِ مایکروسافت نیز استفاده کنید. فقط لازم است این الگو را رعایت کنید. در پیش‌گفتار برخی جزئیات مشخصات این الگو نگاشته شده است، می‌توانید بر مبنای آن پایان‌نامه خود را آماده کنید.

 