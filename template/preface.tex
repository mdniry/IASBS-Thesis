% !TeX root = main.tex
% !TeX spellcheck = fa_IR
\section*{پیش‌گفتار}
\thispagestyle{plain} % Remove header on this specific page
\addcontentsline{toc}{section}{پیش‌گفتار}
الگوی قدیمی صفحه‌آرایی و حروف‌چینی رساله‌ها و پایان‌نامه‌ها براساس امکاناتی نظیر ماشین‌تحریر شکل گرفته بود. به همین سبب، لازم بود صفحات در قطع 
$\mathrm{A4}$، 
یک خط در میان، و یک‌رو حروفچینی شوند و با جلدی ضخیم صحافی شوند تا نگهداری آن‌ها برای مدت طولانی ممکن باشد. با رشد روزافزون فناوری، سال‌ها ست که در بسیاری از دانشگاه‌های دنیا این الگوی قدیمی کنار گذاشته شده است و پایان‌نامه‌ها به مدد امکانات رایانه‌ای و چاپگرهای لیزری با طرحی شبیه کتاب آماده و تدوین می‌شوند.

به همین سبب از مدتی قبل لزوم تدوین و انتشار الگویی که بر پایهٔ فناوری روز بنا شده باشد احساس می‌شد و شورای آموزش دانشگاه تحصیلات تکمیلی علوم‌پایه زنجان نیز با تصویب یک مصوبه و ارائه یک راهنما این نیاز را تایید کرد. الگوی حاضر برمبنای همان راهنما و با مشورت جمعی از اساتید دانشگاه تدوین شده است. تلاش شده تا حد ممکن نکات ظریف بیان شده در آن راهنما در این الگو گنجانده شود. با این وجود توصیه می‌شود قبل از شروع به نگارش 
\thesis 
و اعمال تغییرات در محتوی این الگو، حتماً یکبار آن راهنما را مطالعه کنید.

الگوی حاضر برمبنای کلاس\dash سند 
\lr{`iasbs-thesis.cls'} 
طراحی شده است. در این کلاس علاوه بر گزینه‌های شناخته شدهٔ کلاس\dash سند
\lr{`report'} 
در لاتک می‌توانید از گزینه‌های 
\lr{`phd'}، \lr{`master'}، \lr{`proposal'}، \lr{`review'}، 
نیز استفاده کنید. با ترکیب مناسب این گزینه‌ها می‌توان از این الگو برای نگارش نسخهٔ نهایی، پیش‌نویس، یا پیشنهادِ رسالهٔ دکتری و یاپایان‌نامهٔ کارشناسی ارشد استفاده کرد. در صورت تمایل حتی می‌توانید پروژه کارشناسی را نیز با آن بنویسید. بسته به انتخاب شما و ترکیب گزینه‌های کلاس، عملکرد کلاس تغییر می‌کند و چیدمان، تعداد صفحات ابتدا و انتهایی را تنظیم می‌کند. حالت پیش‌نویس برای تسهیل مرور متن 
\thesis 
برای اساتید راهنما، مشاور و داوران ترتیب صفحات را تغییر می‌دهد و صفحات متناظر فارسی و انگلیسی را پشت سر هم قرار می‌دهد تا مقایسه و تطبیق آنها ساده شود. با حذف این گزینه، ترتیب درست صفحات انتخاب می‌شود. صفحاتی مثل قدردانی و حق تألیف در پیشنهاد پروژه حذف می‌شوند. برای روشن شدن موضوع و همینطور به‌قصد آموزش نحوهٔ ساخت جداول بزرگ در فصل روش‌ها جدولی گنجانده شده است که چیدمان صفحات را براساس گزینه‌های انتخابی نشان می‌دهد، جدول~
\ref{tab3:3}.

این الگو به‌طور پیش‌فرض برای چاپ روی کاغذ 
\lr{A4} 
آماده شده است. این حالت برای پیش‌نویس و پیشنهاد
\thesis 
مناسب است. با این حال برای نسخهٔ نهایی 
\thesis 
می‌توانید آن را در کاغذ 
\lr{B5} 
(با اندازهٔ  
$177\unit{mm} \times 249\unit{mm}$) 
چاپ کنید. برای این کار، بسته به قابلیت‌های چاپگر، می‌توانید صفحات را در مقیاس 
$84\,\%$ 
چاپ کنید یا حاشیه‌های چپ و راست کاغذ 
\lr{A4} 
را از هر طرف 
$16.5\unit{mm}$ 
و حاشیه‌های بالا و پایین را 
$24\unit{mm}$ 
کوچک کنید. پس از صفحات اصلی 
\thesis 
دو صفحهٔ اضافی چاپ می‌شود که در شمارش صفحات لحاظ نمی‌شوند. صفحهٔ اول خطوط برش را برای کاغذ 
\lr{A4} 
نشان می‌دهد تا برش صفحات به اندازهٔ 
\lr{B5} 
ساده شود. صفحهٔ بعد، که آخرین صفحهٔ اضافی است، جلد است. این صفحه باید در کاغذ گلاسه با ابعاد 
\lr{A3} 
(با اندازهٔ 
$297\unit{mm} \times 420\unit{mm}$) 
چاپ شود و سپس از روی خطوط راهنما برش بخورد تا جلد مناسب نسخهٔ
\lr{B5} 
به‌دست آید. لازم به ذکر است که ضخامت عطف براساس کاغذ $80$ گرمی محاسبه می‌شود. عمداً صفحه جلد از طرفین دو میلیمتر بزرگ‌تر در نظر گرفته شده است تا تطبیق لبه‌های صفحات با جلد در برش نهایی ساده‌تر انجام شود.

الزامی نیست که 
\thesis 
شما حتماً با این الگو و به لاتِک نوشته شود؛ کافی است شکل ظاهری این الگو و ترتیب صفحات آن رعایت شود. به‌عنوان مثال، می‌توانید 
\thesis{ٔ} 
خود را با نرم‌افزارهایی نظیر آفیسِ مایکروسافت تهیه کنید. در این الگو در قطع 
\lr{A4} 
اندازه قلم $12$ و حاشیه‌های بالا، راست، پایین و چپ کاغذ به ترتیب 
$40$، $24$، $24$، و $36$ میلی‌متر است. فاصله خطوط می‌بایست به گونه‌ای تنظیم شود که حدود $45$ خط در یک صفحه بگنجد. برای متن فارسی از قلم نیلوفر، برای متن انگلیسی از قلم رومی نوین%
\footnote{قلم‌هایی نظیر \lr{``Times New Roman"}} 
و برای اعداد از قلم یاس با صفر توخالی استفاده شود.

صفحهٔ اول هر فصل، فهرست‌ها و سایر صفحات اصلی می‌بایست شماره فرد داشته باشند. شماره صفحه در اولین صفحهٔ هر فصل در میانه پایین صفحه درج می‌شود و در صفحات بعدی نزدیک به حاشیه میانی و در بالای صفح می‌آید. در خارج حاشیه بالایی صفحات زوج، عنوان 
\thesis 
و در خارج حاشیه بالایی صفحات زوج، عنوان فصل نوشته می‌شود. زیرنویس‌ها تا جایی که ممکن باشد دوستونه تنظیم می‌شوند. اعداد در متن 
\thesis 
و همچنین در روابط ریاضی همه‌جا می‌بایست فارسی باشند. تنها اعدادی که در متن مراجع انگلیسی، کدها و شبه‌کدهایی که به صورت کامل انگلیسی نوشته شوند به انگلیسی درج می‌شود.


\subsubsection{علت نگارش پیش‌گفتار}
در شرایط عادی اگر فصل مقدمه را خوب بنویسید نیازی به نگارش پیشگفتار ندارید و می‌توانید آن را حذف کنید. اما خصوصاً در موضوعات بین رشته‌ای یا وقتی پژوهش شما شامل بخش‌های متنوع تئوری، تجربی و محاسباتی است، ساختار 
\thesis 
ممکن است پیچیده‌تر از حالت معمول باشد. در این موارد، حتی ممکن است مجبور شوید بیش از یک فصل مقدمه داشته باشید. در چنین شرایطی بهتر است پیشگفتار نوشته شود. 

در پیشگفتار می‌توانید مفصل‌تر از چکیده، مسأله را شرح دهید و علت ساختار پیچیده 
\thesis 
را روشن کنید. در فصول بعدی شما نمی‌توانید قبل از بیان یک مفهوم از آن استفاده کنید و ملزم به رعایت سیر منطقی هستید. بنابراین ممکن است بحث‌های مفصل و گسترده‌ای لازم باشد تا مقدمات بیان مسأله فراهم شود. در پیشگفتار لازم نیست مسأله به‌طور دقیق بیان شود، به همین دلیل نیاز به مقدمه‌چینی کمتری است. از طرفی، با طرح مساله در پیشگفتار، می‌توان مسیر و ساختار 
\thesis 
را بهتر بیان کرد و ذهن خواننده را برای دنبال کردن مقدمات آماده و توجیه کرد.

در اینجا لازم می‌دانم از تمامی دوستان و همکارانی که با مطالعه متن الگوی حاضر در رفع ایرادها و تکمیل آن مشارکت کردند، تا الگویی مناسب برای دانشجویان دانشگاه فراهم شود، تقدیر و تشکر کنم. همچنین سپاسگزار خواهم شد که پیشنهادات و مشکلات این الگو را که ضمن استفاده از آن  مشخص خواهد شد با ایمیل%
\LTRfootnote{\href{mailto:m.d.niry@iasbs.ac.ir}{m.d.niry@iasbs.ac.ir}} 
برایم ارسال فرمایید.\\[1em]
محمّد دهقان نیری\\
هیات علمی دانشــکده فیزیک\\
زمسـتان ۱۴۰۳
