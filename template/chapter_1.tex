% !TeX root = main.tex
% !TeX spellcheck = fa_IR
\chapter{مقدمه}
در نگارش این الگو تعمداً سعی شده است علاوه بر ارائه توضیحات آموزشی از ابزار‌های مختلف 
\xepersian 
و 
\latex 
استفاده شود. شما می‌توانید این موارد را به عنوان مثال ببینید و استفاده کنید. 

معمولا ساختار \thesis به چند فصل تقسیم می‌شود. محتوی و نام‌گذاری این فصول سلیقه‌ای است. یکی از انتخاب‌ها به شرح زیر است،
\begin{enumerate}
\item
مقدمه، 
\item 
تاریخچه، 
\item 
مواد و روش‌ها (الگوریتم‌ها و روش‌ها)، 
\item 
نتایج، 
\item 
بحث و جمع‌بندی.
\end{enumerate}
شما می‌توانید تاریخچه را به‌طور مختصر در مقدمه بیان کنید و فصل تاریخچه را حذف کنید. کافیست خط
\begin{latin}\begin{verbatim}
% !TeX root = main.tex
% !TeX spellcheck = fa_IR
\chapter{تاریخچه پژوهش}

در این بخش به معرفی کلی پژوهش‌های انجام شده در زمینه مورد نظر پرداخته می‌شود. هدف اصلی این فصل، ارائه تصویری جامع از پیشینه مسأله و پژوهش است. ممکن است به جای این فصل قسمتی از مقدمه به بیان تاریخچه مسأله اختصاص یابد و این فصل حذف شود.

این فصل می‌تواند با بررسی کارهای مرتبط پژوهشی که در گروه پژوهشی که در آن عضویت دارید آغاز شود. سپس به بررسی پژوهش‌های مرتبط که داخل کشور انجام شده بپردازید و نهایتا با مرور مسأله‌های مرتبط که در مقالات گروه‌های بین‌المللی گزارش شده خاتمه یابد.

میزان شرح و بسط تاریخچه تا حدی به سلیقه شما و موضوع و مسأله‌ای که روی آن کار می‌کنید بستگی دارد. استاد راهنما می‌تواند در این مورد بهترین راهنمایی را به شما بدهد.

\section{جمع‌بندی}
در این قسمت، نتایج مرور پژوهش‌های انجام شده خلاصه و تحلیل می‌شوند. این تحلیل می‌تواند به شناسایی نقاط قوت و ضعف تحقیقات قبلی و تعیین شکاف‌های پژوهشی که قرار است پروژه پژوهشی شما برخی از آنها را پر کند، کمک کند.


\end{verbatim}
\end{latin}\noindent
را در 
\lr{main.tex}
غیرفعال کنید تا فصل تاریخچه حذف شود. ممکن است بسته به موضوع، ترتیب فصول تغییر داده شود یا فصول جدیدی اضافه شود. در انجام این کار با هدایت استاد راهنما آزاد هستید. هدف نگارش 
\thesis 
به‌نحوی است که برای خواننده درک و دنبال کردن آن ساده‌تر باشد. در واقع، مهمترین هدف از نگارش هر متن علمی انتقال حداکثر مفاهیم است. 

در هر فصل، یک قالب پیشنهادی برای بخش‌بندی فصل ارائه شده است که کاملاً سلیقه‌ای است و شما در تغییر آن بسته به موضوع و مسأله خود کاملاً آزاد هستید. علاوه بر آن در فصل روش‌ها نحوه نگارش 
\thesis 
و استفاده از 
\latex 
برای ایجاد شکل و جدول توضیح داده شده است. 

در ادامه ساختار پیشنهادی فصل مقدمه را می‌آوریم. مقدمهٔ 
\thesis
باید به سوالاتی پاسخ دهد. لازم نیست حتماً جواب هر سوال در یک بخش داده شود. ممکن است جواب برخی از این سوالات را در یک پاراگراف بیاورید. بنابراین می‌توانید این بخش‌بندی را به سلیقه خودتان تغییر دهید و بعضی موارد را با هم ترکیب کنید. هدف این است که سوالاتی که قرار است در مقدمه به آنها پاسخ دهید یکجا یادآوری شوند:


\section{زمینه و اهمیت موضوع}
مقدمه می‌بایست زمینهٔ پژوهش را توضیح دهد و اهمیت و ارزش موضوع پژوهش را به وضوح بیان کند. چرا این مسأله مهم است و حل آن چه تاثیری می‌تواند در حوزهٔ علمی مورد نظر داشته باشد؟


\section{مسألهٔ پژوهش}
در هر پژوهشی مسألهٔ می‌بایست به‌طور دقیق و واضح تعریف شود. بهترین نقطه برای انجام این کار مقدمه است. باید مشکل یا چالشی که قصد دارید در این پژوهش حل کنید را بیان کنید. ممکن است در ابتدای مقدمه نتوانید تعریف دقیق مسأله را بیان کنید. اما بعد از بیان مفاهیم می‌توانید مسأله را به‌صورت دقیق بیان کنید.


\section{اهداف پژوهش}
اهداف کلی و جزئی پژوهش خود را مشخص کنید. اهداف باید واقع‌بینانه، قابل دستیابی و مرتبط با مسألهٔ باشند. این قسمت به‌نحوی با بیان مسأله مرتبط است و ممکن است بخواهید هر دو را در قالب یک بخش بیان کنید.


\section{سوالات یا فرضیات پژوهش}
در این قسمت، دانشجو باید سوالات اصلی و فرعی یا فرضیات را مطرح کند. این سوالات یا فرضیات باید به گونه‌ای باشند که بتوان آن‌ها را در طول پژوهش پاسخ داد یا آزمایش کرد.


\section{ساختار رساله/پایان‌نامه}
اگر از قالب شناخته شده فصل‌بندی استفاده نمی‌کنید حتما باید به‌طور مختصر به ساختار کلی 
\thesis 
اشاره کنید و توضیح دهد که هر فصل شامل چه مباحثی می‌باشد. اما ممکن است ترجیح دهید این کار را در پیشگفتار انجام دهید.


\section{محدودیت‌های پژوهش}
با مطالعه تاریخچه مسأله متوجه محدودیت‌ها و چالش‌هایی می‌شوید که در حل مسأله با آن‌ها مواجه خواهید شد. می‌توانید به چنین مواردی در فصل مقدمه یا بسته به مورد در فصل تاریخچه اشاره کنید.


\section{تعریف اصطلاحات}
حل هر مسأله به شناخت مفاهیمی متکی است که ممکن است جزء دانش عمومی خوانندگان نباشد. مقدمه حتماً باید شامل بخشی باشد که اصطلاحات کلیدی مورد استفاده در پژوهش را تعریف کند تا خواننده با مفاهیم مورد بحث آشنا شود.

رعایت این ساختار الزامی نیست، اما با بیان همه موارد ذکر شده، مقدمه‌ای جامع و کامل برای 
\thesis 
تهیه می‌شود که به خواننده کمک می‌کند تا با زمینهٔ پژوهش، اهمیت آن، اهداف و سوالات مطرح، و ساختار کلی 
\thesis
آشنا شود.
