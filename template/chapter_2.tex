% !TeX root = main.tex
% !TeX spellcheck = fa_IR
\chapter{تاریخچه پژوهش}

در این بخش به معرفی کلی پژوهش‌های انجام شده در زمینه مورد نظر پرداخته می‌شود. هدف اصلی این فصل، ارائه تصویری جامع از پیشینه مسأله و پژوهش است. ممکن است به جای این فصل قسمتی از مقدمه به بیان تاریخچه مسأله اختصاص یابد و این فصل حذف شود.

این فصل می‌تواند با بررسی کارهای مرتبط پژوهشی که در گروه پژوهشی که در آن عضویت دارید آغاز شود. سپس به بررسی پژوهش‌های مرتبط که داخل کشور انجام شده بپردازید و نهایتا با مرور مسأله‌های مرتبط که در مقالات گروه‌های بین‌المللی گزارش شده خاتمه یابد.

میزان شرح و بسط تاریخچه تا حدی به سلیقه شما و موضوع و مسأله‌ای که روی آن کار می‌کنید بستگی دارد. استاد راهنما می‌تواند در این مورد بهترین راهنمایی را به شما بدهد.

\section{جمع‌بندی}
در این قسمت، نتایج مرور پژوهش‌های انجام شده خلاصه و تحلیل می‌شوند. این تحلیل می‌تواند به شناسایی نقاط قوت و ضعف تحقیقات قبلی و تعیین شکاف‌های پژوهشی که قرار است پروژه پژوهشی شما برخی از آنها را پر کند، کمک کند.

